% Definierar dokumentformat
\documentclass[a4paper, 11pt]{article}

% Början på dokument
\begin{document}

% Titel
\title{\LaTeX{} Test - Varmland}
\author{Author: Niklas Hokenstrom (Wikipedia)}

\maketitle

% Rubrik
\section{Introduction}

Varmland is a historical province or landskap in the west of middle Sweden. It borders Vastergotland, Dalsland, Dalarna, Vastmanland and Narke. It is also bounded by Norway in the west. Latin name versions are Vermelandia and Wermelandia. Although the province's land originally was Gotaland, the province's current land is Svealand.

Varmland is a historical province or landskap in the west of middle Sweden. It borders Vastergotland, Dalsland, Dalarna, Vastmanland and Narke. It is also bounded by Norway in the west. Latin name versions are Vermelandia and Wermelandia. Although the province's land originally was Gotaland, the province's current land is Svealand. The origin of the province's name is uncertain. It may refer to the large local lake by the name of Varmeln, although the lake's name is parsed as var- + -meln (not varm 'heat' + eln). Ny Rad

% Rubrik
\section{Administration}
Provinces serve no administrative function in Sweden today. Instead, that function is served by Counties of Sweden; however, in many instances a county has virtually the same border as the province, which is the case with Varmland and its corresponding county Varmland County. The main exception is a smaller part to the south east which belongs to Orebro County.

% Rubrik
\section{Heraldry}
Arms were granted in 1560, when it depicted a wolverine. This was however too similar to that of Medelpad. In 1567 it was revised into an eagle. In the late 17th century the eagle was black. In 1936 it got its current blazon, when the eagle became blue. Blazon: "Argent, an Eagle displayed Azure beaked, langued and membered Gules."

% Rubrik
\section{Geography}
The largest lake is Vanern. Most streams of importance lead to Vanern. However, the province is rich in small lakes, ponds and streams. The scenic nature with mountains and lakes is usually regarded among the most picturesque in Sweden, and has inspired painters and writers since the 19th century.

% Underrubrik
\subsection{Western Varmland}
There are several mountain plateaus in the western part of Varmland, which is in the Scandinavian mountain range. The highest elevations are found in the northern parts, with plateaus of 500–700 meters. Here also the highest mountain top is found, the Granberget at Holjes, 701 meters.

% Underrubrik
\subsection{Eastern Varmland}
The eastern part of Varmland is counted into the Bergslagen, the Central Swedish Mining District. Its terrain is rather hilly, but a few high-altitude hills are present: Hvitklinten (414 m.), Dalkarlsberget (450 m.) and Valbergsros (476 m.).

This part of Varmland is rich in minerals, most notably iron ore which exists in large quantities. Some notable sites in this area are around Langban and Nordmark Hundred. In the southeast, the ridge of Kilsbergen marks the border with Narke.

% Slut på dokument
\end{document}